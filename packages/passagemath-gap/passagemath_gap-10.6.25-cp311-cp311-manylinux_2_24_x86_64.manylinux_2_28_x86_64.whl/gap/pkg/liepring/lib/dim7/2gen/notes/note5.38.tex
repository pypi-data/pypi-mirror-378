
\documentclass[12pt]{article}
%%%%%%%%%%%%%%%%%%%%%%%%%%%%%%%%%%%%%%%%%%%%%%%%%%%%%%%%%%%%%%%%%%%%%%%%%%%%%%%%%%%%%%%%%%%%%%%%%%%%%%%%%%%%%%%%%%%%%%%%%%%%%%%%%%%%%%%%%%%%%%%%%%%%%%%%%%%%%%%%%%%%%%%%%%%%%%%%%%%%%%%%%%%%%%%%%%%%%%%%%%%%%%%%%%%%%%%%%%%%%%%%%%%%%%%%%%%%%%%%%%%%%%%%%%%%
\usepackage{amsfonts}
\usepackage{amssymb}
\usepackage{sw20elba}

%TCIDATA{OutputFilter=LATEX.DLL}
%TCIDATA{Version=5.50.0.2890}
%TCIDATA{<META NAME="SaveForMode" CONTENT="1">}
%TCIDATA{BibliographyScheme=Manual}
%TCIDATA{Created=Thursday, June 27, 2013 17:23:01}
%TCIDATA{LastRevised=Sunday, August 04, 2013 16:00:12}
%TCIDATA{<META NAME="GraphicsSave" CONTENT="32">}
%TCIDATA{<META NAME="DocumentShell" CONTENT="Articles\SW\mrvl">}
%TCIDATA{CSTFile=LaTeX article (bright).cst}
%TCIDATA{ComputeDefs=
%$A=\left( 
%\begin{array}{cc}
%t & x \\ 
%y & z%
%\end{array}%
%\right) $
%$P=\left( 
%\begin{array}{ll}
%-1 & 0 \\ 
%0 & 1%
%\end{array}%
%\right) $
%}


\newtheorem{theorem}{Theorem}
\newtheorem{axiom}[theorem]{Axiom}
\newtheorem{claim}[theorem]{Claim}
\newtheorem{conjecture}[theorem]{Conjecture}
\newtheorem{corollary}[theorem]{Corollary}
\newtheorem{definition}[theorem]{Definition}
\newtheorem{example}[theorem]{Example}
\newtheorem{exercise}[theorem]{Exercise}
\newtheorem{lemma}[theorem]{Lemma}
\newtheorem{notation}[theorem]{Notation}
\newtheorem{problem}[theorem]{Problem}
\newtheorem{proposition}[theorem]{Proposition}
\newtheorem{remark}[theorem]{Remark}
\newtheorem{solution}[theorem]{Solution}
\newtheorem{summary}[theorem]{Summary}
\newenvironment{proof}[1][Proof]{\noindent\textbf{#1.} }{{\hfill $\Box$ \\}}
\input{tcilatex}
\addtolength{\textheight}{30pt}

\begin{document}

\title{Algebra 5.38}
\author{Michael Vaughan-Lee}
\date{June 2013}
\maketitle

Algebra 5.38 has$\allowbreak $ $\gcd (p-1,3)(p^{2}+3p+10)+p+6$ immediate
descendants of order $p^{7}$. Of these $\frac{1}{2}((p^{2}+3p+11)\gcd
(p-1,3)+1)$ come from one 4 parameter family of algebras, and $\frac{1}{2}%
(\gcd (p-1,3)(p^{2}+p+1)+5)$ come from another four parameter family. In
both cases we take the four parameters as entries in a $2\times 2$ matrix%
\[
A=\left( 
\begin{array}{cc}
x & y \\ 
z & t%
\end{array}%
\right) , 
\]%
and in both cases we consider the orbits of matrices $A$ of this form over GF%
$(p)$ under an action of the subgroup of GL$(2,p)$ consisting of
non-singular matrices of the form%
\[
\left( 
\begin{array}{ll}
\alpha & \beta \\ 
\beta & \alpha%
\end{array}%
\right) \text{ or }\left( 
\begin{array}{ll}
\alpha & \beta \\ 
-\beta & -\alpha%
\end{array}%
\right) . 
\]

In the first case two matrices $A$ and $B$ give isomorphic Lie rings if and
only if%
\[
B=\left( 
\begin{array}{ll}
\alpha  & \beta  \\ 
\beta  & \alpha 
\end{array}%
\right) A\left( 
\begin{array}{ll}
(\alpha ^{4}-\beta ^{4}) & 2\alpha \beta (\alpha ^{2}-\beta ^{2}) \\ 
2\alpha \beta (\alpha ^{2}-\beta ^{2}) & \alpha ^{4}-\beta ^{4}%
\end{array}%
\right) ^{-1}
\]%
or 
\[
B=\left( 
\begin{array}{ll}
\alpha  & \beta  \\ 
-\beta  & -\alpha 
\end{array}%
\right) A\left( 
\begin{array}{ll}
-(\alpha ^{4}-\beta ^{4}) & -2\alpha \beta (\alpha ^{2}-\beta ^{2}) \\ 
2\alpha \beta (\alpha ^{2}-\beta ^{2}) & \alpha ^{4}-\beta ^{4}%
\end{array}%
\right) ^{-1}
\]%
for some $\alpha ,\beta $. In the second case, two matrices $A$ and $B$ give
isomorphic Lie rings if and only if%
\[
B=\left( 
\begin{array}{ll}
\alpha  & \beta  \\ 
\omega \beta  & \alpha 
\end{array}%
\right) A\left( 
\begin{array}{ll}
\alpha ^{4}-\omega ^{2}\beta ^{4} & 2\alpha \beta (\alpha ^{2}-\omega \beta
^{2}) \\ 
2\omega \alpha \beta (\alpha ^{2}-\omega \beta ^{2}) & \alpha ^{4}-\omega
^{2}\beta ^{4}%
\end{array}%
\right) ^{-1}
\]%
or 
\[
B=\left( 
\begin{array}{ll}
\alpha  & \beta  \\ 
-\omega \beta  & -\alpha 
\end{array}%
\right) A\left( 
\begin{array}{ll}
-(\alpha ^{4}-\omega ^{2}\beta ^{4}) & -2\alpha \beta (\alpha ^{2}-\omega
\beta ^{2}) \\ 
2\omega \alpha \beta (\alpha ^{2}-\omega \beta ^{2}) & \alpha ^{4}-\omega
^{2}\beta ^{4}%
\end{array}%
\right) ^{-1}
\]%
for some $\alpha ,\beta $.

A simple loop over all possible $A$ and all possible $\alpha ,\beta $ can
find representatives for the orbits. You can shorten the search slightly by
noting that in both cases if we take $\alpha =-1$, $\beta =0$ then $B=-A$.

\end{document}
